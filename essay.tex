\documentclass[a4j,10pt, twocolumn]{jarticle}
\usepackage[dvipdfmx]{graphicx}
\usepackage{amssymb}
\usepackage{amsmath}
\usepackage{float}
\usepackage{slashbox}
%---------------------------------------------------
% ページの設定
%---------------------------------------------------
\setlength{\textwidth}{170truemm}
\setlength{\textheight}{250truemm}
\setlength{\topmargin}{-14.5truemm}
\setlength{\oddsidemargin}{-5.5truemm}
\pagestyle{empty}
\setlength{\headheight}{0truemm}
\setlength{\parindent}{1zw}

\begin{document}
\twocolumn
[
\begin{center}
  {\huge NAISTにて取り組む研究について}
\end{center}
\begin{flushright}
\begin{tabular}{ll}
氏名: & 新妻巧朗\\
試験区分: & 情報科学区分\\
希望研究室: & 自然言語処理研究室\\
\end{tabular}
\end{flushright}
\vspace{2truemm}
]
\section{はじめに}
私はNAISTにて「情報探索における特徴表現の手法」について研究をしたいと考えている。
\section{研究の背景}
現代の検索エンジンには、情報検索スキルを要求するという課題がある。検索エンジンに合わせて、直面している問題を情報要求に変換する能力が必要になる。この能力は得意不得意が存在し、結果的に情報格差を生んでいる。教育によって、情報検索スキルを向上させるための研究[1]も存在しておりますが、私は検索エンジンが解決すべき課題だと考えています。
\section{研究の展望}
情報検索システムにおいて、文書と検索語の比較アルゴリズムには、TF-IDFモデルや確率モデル、ベクトル空間モデルをはじめにいくつか存在している。
これらのモデルは、文書と検索語の直接的な比較が中心?
私は自然言語理解技術に焦点を当てて、文書のコンテキストを反映した
情報検索システムは巨大で数多くの要素技術から構成される。そのため、自然言語理解に焦点を当てる。自然言語理解の技術
brabra
\section{まとめ}
brabra
\begin{thebibliography}{3}
\bibitem{3} \label{takada}
著者名(年),``論文タイトル", なんの学会誌200X,(SMB YYYY).
\end{thebibliography}
\end{document}
