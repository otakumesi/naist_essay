\documentclass[a4j,10pt, twocolumn]{jarticle}
\usepackage[dvipdfmx]{graphicx}
\usepackage{amssymb}
\usepackage{amsmath}
\usepackage{float}
\usepackage{slashbox}
%---------------------------------------------------
% ページの設定
%---------------------------------------------------
\setlength{\textwidth}{170truemm}
\setlength{\textheight}{250truemm}
\setlength{\topmargin}{-14.5truemm}
\setlength{\oddsidemargin}{-5.5truemm}
\pagestyle{empty}
\setlength{\headheight}{0truemm}
\setlength{\parindent}{1zw}

\begin{document}
\twocolumn
[
\begin{center}
  {\huge NAISTにて取り組む研究について}
\end{center}
\begin{flushright}
\begin{tabular}{ll}
氏名: & 新妻巧朗\\
試験区分: & 情報科学区分\\
希望研究室: & 自然言語処理研究室\\
\end{tabular}
\end{flushright}
\vspace{2truemm}
]
\section{はじめに}
私はNAISTにて、「情報検索システムにおけるファセットの自動生成」について研究をしたいと考えている。
\section{研究の背景}
\subsection{研究の社会的意義}
現代社会において情報収集をおこなうためには、検索エンジンを利用することは必要不可欠である。しかし、検索エンジンを適切に活用できずに、目的の情報に至れない人も存在する。それは現代の検索エンジンの仕組みが、利用者に対して情報検索能力を要求しているからだ。この能力は個人差が大きく、能力差によって情報格差が生じていることがわかっている[1]。こうした課題を解決することによって、情報に辿りつけないことで引き起こされている機会損失を減らすことができる。教育を通して情報検索能力を向上させる研究[2]も存在しているが、本研究ではシステムによって解決するアプローチを考えていく。
\subsection{研究の背景}
言語能力の高さが情報検索能力に高さに関係しているとされている[1]。つまり、言語能力の高低が情報検索における情報格差を生み出していると考えられる。そのため、情報の探索過程にて言語能力を要求する場面で、その能力を補助をするシステムを提案することで解決をしたいと考えている。

情報の探索行動を掘り下げると、検索意図と検索対象の距離を縮めていくプロセスと考えられる。
そのため、私は検索システムのファセット検索が役に立つのではないかと考えている。ファセット検索とはetc\dots。
つまり、検索システムの利用者の代わりに検索の意図を掘り下げて言語化をする機能を提供しているとも言えるからだ。

そこで、検索キーワードに対して適切なファセットを提供し、インタラクティブに検索意図を読み取るシステムを考えている。
\subsection{問題の解決法の提案}
\section{研究の展望}

\section{まとめ}
brabra
\begin{thebibliography}{3}
\bibitem{3} \label{takada}
  [1] 福島健介・小原 格・須原慎太郎・生田 茂(2005), "インターネット検索能力の差異に及ぼす 要因の検討 その1", コンピュータ&エデュケーション VOL.18 2005
  [2] 齋藤ひとみ・三輪和久(2004),  "Web 情報検索におけるリフレクションの支援", 人工知能学会論文誌 19 巻 4 号 C(2004 年)
\end{thebibliography}
\end{document}
